\documentclass[11pt]{article}
\usepackage{amsmath,amssymb,amsthm}
\usepackage{graphicx}
\usepackage[margin=1in]{geometry}
\usepackage{fancyhdr}
\setlength{\parindent}{0pt}
\setlength{\parskip}{5pt plus 1pt}
\setlength{\headheight}{13.6pt}
\newcommand\question[2]{\vspace{.25in}\hrule\textbf{#1: #2}\vspace{.5em}\hrule\vspace{.10in}}
\renewcommand\part[1]{\vspace{.10in}\textbf{(#1)}}
\newcommand\algorithm{\vspace{.10in}\textbf{Algorithm: }}
\newcommand\result{\vspace{.10in}\textbf{Result: }}
\pagestyle{fancyplain}
\lhead{\textbf{\NAME\ (\ANDREWID)}}
\chead{\textbf{Assignment\HWNUM}}
\rhead{STAT3006: Statistical Computing}
\begin{document}\raggedright
%Section A==============Change the values below to match your information==================
\newcommand\NAME{ZHANG Xinfang}  % your name
\newcommand\ANDREWID{1155141566}     % your student id
\newcommand\HWNUM{1}              % the homework number
%Section B==============Put your answers to the questions below here=======================

\question{1}{The Bisection Method (25\%)} 
For function $f(x) = x^3 + 6x^2 + \pi x - 12$, the derivative is $f'(x) = 3x^2 + 12x + \pi$. The we can calculate that 
zeros of the derivative are $\frac{-12 - \sqrt{12(12-\pi)}}{6}$ and $\frac{-12 + \sqrt{12(12-\pi)}}{6}$.

$f(\frac{-12 - \sqrt{12(12-\pi)}}{6}) = 7.864841$ and $f(\frac{-12 + \sqrt{12(12-\pi)}}{6}) = -12.43121$
Hence, the function $f$ has totally 3 zeros.

\algorithm{Bisection Method in the R file.}

\result{zeros -4.837944, -2.259727, and 1.097664.}

\question{2}{Poisson Regression - Newton's Method (25\%)}

\part{1} Since $y_i \sim Poisson(\lambda_i)$ and $log(\lambda_i) = \alpha + \beta x_i + \gamma x_i^2$, we can get the Likelihhod function:

\begin{align*}
    L(\alpha, \beta, \gamma|\mathbf{x}, \mathbf{y}) = \prod_{i=1}^{n} \frac{\lambda_i^{y_i} e^{-\lambda_i}}{y_i !} = \prod_{i=1}^{n} \frac{e^{(\alpha + \beta x_i + \gamma x_i^2)y_i} e^{-e^{\alpha + \beta x_i + \gamma x_i^2}}}{y_i !}
\end{align*}

\part{2} The log-Likelihood function is 
\begin{align*}
    l(\alpha, \beta, \gamma|\mathbf{x}, \mathbf{y}) = \sum_{i=1}^{n} (\alpha + \beta x_i + \gamma x_i^2)y_i - e^{\alpha + \beta x_i + \gamma x_i^2} - \log y_i !  
\end{align*}
Then we have:
\begin{align*}
    \frac{\partial l(\alpha, \beta, \gamma|\mathbf{x}, \mathbf{y})}{\partial \alpha}\bigg\rvert_{\hat{\alpha}} &= \sum_{i=1}^{n} [y_i - e^{\alpha + \beta x_i + \gamma x_i^2}]\bigg\rvert_{\hat{\alpha}} = 0\\
    \frac{\partial l(\alpha, \beta, \gamma|\mathbf{x}, \mathbf{y})}{\partial \beta}\bigg\rvert_{\hat{\beta}} &= \sum_{i=1}^{n} [x_iy_i - x_ie^{\alpha + \beta x_i + \gamma x_i^2}]\bigg\rvert_{\hat{\beta}} = 0\\
    \frac{\partial l(\alpha, \beta, \gamma|\mathbf{x}, \mathbf{y})}{\partial \gamma}\bigg\rvert_{\hat{\gamma}} &= \sum_{i=1}^{n} [x_i^2y_i - x_i^2e^{\alpha + \beta x_i + \gamma x_i^2}]\bigg\rvert_{\hat{\gamma}} = 0\\
\end{align*}
Let $\mathbf{F}(\mathbf{x}) = \begin{pmatrix}
    \frac{\partial l(\alpha, \beta, \gamma|\mathbf{x}, \mathbf{y})}{\partial \alpha}\\\frac{\partial l(\alpha, \beta, \gamma|\mathbf{x}, \mathbf{y})}{\partial \beta}\\\frac{\partial l(\alpha, \beta, \gamma|\mathbf{x}, \mathbf{y})}{\partial \gamma}
\end{pmatrix}$, then $\mathbf{F'}(\mathbf{x}) = \begin{pmatrix}
    \frac{\partial^2 l(\alpha, \beta, \gamma|\mathbf{x}, \mathbf{y})}{\partial \alpha^2} && \frac{\partial^2 l(\alpha, \beta, \gamma|\mathbf{x}, \mathbf{y})}{\partial \beta \partial \alpha} && \frac{\partial^2 l(\alpha, \beta, \gamma|\mathbf{x}, \mathbf{y})}{\partial \gamma \partial \alpha}\\
    \frac{\partial^2 l(\alpha, \beta, \gamma|\mathbf{x}, \mathbf{y})}{\partial \alpha \partial \beta} && \frac{\partial^2 l(\alpha, \beta, \gamma|\mathbf{x}, \mathbf{y})}{\partial \beta^2} && \frac{\partial^2 l(\alpha, \beta, \gamma|\mathbf{x}, \mathbf{y})}{\partial \gamma \partial \beta}\\
    \frac{\partial^2 l(\alpha, \beta, \gamma|\mathbf{x}, \mathbf{y})}{\partial \alpha \partial \gamma} && \frac{\partial^2 l(\alpha, \beta, \gamma|\mathbf{x}, \mathbf{y})}{\partial \beta \partial \gamma} && \frac{\partial^2 l(\alpha, \beta, \gamma|\mathbf{x}, \mathbf{y})}{\partial \gamma^2}
\end{pmatrix}$, in which 
\begin{flalign*}
    \frac{\partial^2 l(\alpha, \beta, \gamma|\mathbf{x}, \mathbf{y})}{\partial \alpha^2} &= \sum_{i=1}^{n} -e^{\alpha + \beta x_i + \gamma x_i^2}\\
    \frac{\partial^2 l(\alpha, \beta, \gamma|\mathbf{x}, \mathbf{y})}{\partial \alpha \partial \beta} &= \frac{\partial^2 l(\alpha, \beta, \gamma|\mathbf{x}, \mathbf{y})}{\partial \beta \partial \alpha} = \sum_{i=1}^{n} -x_ie^{\alpha + \beta x_i + \gamma x_i^2}\\
    \frac{\partial^2 l(\alpha, \beta, \gamma|\mathbf{x}, \mathbf{y})}{\partial \alpha \partial \gamma} &= \frac{\partial^2 l(\alpha, \beta, \gamma|\mathbf{x}, \mathbf{y})}{\partial \gamma \partial \alpha} = \sum_{i=1}^{n} -x_i^2e^{\alpha + \beta x_i + \gamma x_i^2}\\
    \frac{\partial^2 l(\alpha, \beta, \gamma|\mathbf{x}, \mathbf{y})}{\partial \beta^2} &= \sum_{i=1}^{n} -x_i^2e^{\alpha + \beta x_i + \gamma x_i^2}\\
    \frac{\partial^2 l(\alpha, \beta, \gamma|\mathbf{x}, \mathbf{y})}{\partial \beta \partial \gamma} &= \frac{\partial^2 l(\alpha, \beta, \gamma|\mathbf{x}, \mathbf{y})}{\partial \gamma \partial \beta} = \sum_{i=1}^{n} -x_i^3e^{\alpha + \beta x_i + \gamma x_i^2}\\
    \frac{\partial^2 l(\alpha, \beta, \gamma|\mathbf{x}, \mathbf{y})}{\partial \gamma^2} &= \sum_{i=1}^{n} -x_i^4e^{\alpha + \beta x_i + \gamma x_i^2}
\end{flalign*}
Therefore, by Newton's Methhod, given initial guess $\alpha^{(0)}$, $\beta^{(0)}$, and $\gamma^{(0)}$, for each iteration: 
\begin{align*}
    \begin{pmatrix} \alpha^{(n)} \\ \beta^{(n)} \\ \gamma^{(n)} \end{pmatrix} = \begin{pmatrix} \alpha^{(n-1)} \\ \beta^{(n-1)} \\ \gamma^{(n-1)} \end{pmatrix} - \mathbf{F'}[(\mathbf{x})]^{-1}\mathbf{F}(\mathbf{x}).
\end{align*}
Hence, for the algorithm: $\mathbf{x^{(n)}} = \begin{pmatrix} \alpha^{(n)} \\ \beta^{(n)} \\ \gamma^{(n)} \end{pmatrix}$,

STEP 1: Solve $\mathbf{F'}(\mathbf{x^{(n)}}) \mathbf{\triangle x^{(n)}} = - \mathbf{F}(\mathbf{x^{(n)}})$;

STEP 2: Update by $\mathbf{x^{(n+1)}} = \mathbf{x^{(n)}} + \mathbf{\triangle x^{(n)}}$.

\part{3} \algorithm{Newton's Method code in the R file.}

\result{$\alpha = 1.503533$, $\beta = 1.052351$, and $\gamma = 1.957396$.}

\question{3}{Logistic Regression - Newton's Method (20\%)}

\part{1} 

\part{2}

\part{3}

\question{4}{EM Algorithm (30\%)}

\part{1} 

\part{2}

\part{3}

\part{4}

\end{document}